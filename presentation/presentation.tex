\documentclass[10pt]{article}
\author{James Sturgis}
\title{A whole organism metabolic model for LUCA}
\begin{document}
\maketitle
\begin{abstract}

\end{abstract}

\section*{Introduction}
Generally why do we want to know about LUCA?

What are whole organism metabolic models and what are they good for or why are they useful?

What is LUCA: organism, meta-organism or partial-meta-organism?

Where did LUCA live, in a rich or poor environment?

How did LUCA survive?

In this work we have attempted to propose a whole organism metabolic model for LUCA, and used both flux-balance and thermodynamic analyses to understand how the metabolism of this meta-organism might work.

\section*{Resultats}

We undertook to develop a whole organism metabolic model of the last universal common ancestor LUCA. This model, like metabolism today, contains two branches, an anabolic branch allowing synthesis of the cellular content from raw materials available from the environment, and a catabolic branch providing the energy to drive the anabolism.

\subsection*{Defining our ideas of LUCA}

We expect LUCA to share many of the attributes common to all living organisms today, and as such we have accepted a certain number of biochemical characteristics.

% This info should be in a referenced table
\begin{itemize}
 \item A anabolism largely found in organisms today allowing the synthesis of:
 \begin{itemize}
     \item DNA for the genome containing the 4 bases found today (A, T, G, and C)
     \item RNA for protein synthesis as mRNA, tRNA and rRNA containing the 4 common bases found today (A, U, G, and C)
     \item proteins fabricated using the universal genetic code from the 20 amino-acids commonly found in proteins today.
     \item a lipid bilayer membrane capable of chemiosmotic coupling.
     \item the different cofactors and coenzymes required for the enzymes catalysing the above reactions.
 \end{itemize}
 \item A catabolism capable of harvesting energy and reducing power from the environment necessary to drive the anabolic reactions.
 \item an intracellular environment rich in K$^+$  but poor in Na$^+$ at about neutral pH, and an extracellular environment able to provide all the necessary trace elements as well as sources of the main elements C,N,O,S and P.

\end{itemize}

As there is considerable debate about the catabolism and environment of LUCA we chose to develop first an anabolic model and then examine how it functions to produce biomass when driven by different anabolic schemes.

\subsection*{A functional anabolic model}
The anabolic model we have produced is based on that presented by Wimmers \textit{et al.} \cite{Wimmers} incorporating a number of additions as detailed below.
The an overview of this model is shown in figure \ref{fig:anabolism} with the fluxes through the central metabolism and towards the different cellular components.

Description of model goes here number of species and reactions and exchanges.

To construct this model various additions to that presented by Wimmer \textit{et al.} \cite{Wimmers} were necessary. These were:
\begin{itemize}
 \item a BIOMASS reaction to define what LUCA is made out of.
 \item Aggregation and polymerisation reactions to define the biomass components: DNA, RNA, Protein, Membrane and metabolites.
 \item Reactions to make a lipid membrane.
 \item Exchange reactions to define environmental requirements.
 \item Miscellaneous reactions necessary to make the metabolic model functional.
\end{itemize}
These additional reactions aim to extend the model, in as parsimonious way as possible, to give a functional whole organism metabolic model.
\subsubsection*{BIOMASS}
\subsubsection*{Aggregation}
\subsubsection*{Lipid metabolism}
\subsubsection*{Exchange reactions}
\subsubsection*{Miscellaneous additions}
\begin{itemize}
 \item Nucleotide kinase reactions to regenerate CTP from CMP. This is necessary as the membrane lipid metabolism adds headgroups via CDP-diacyl glycerol or CDP-Archeol resulting in the production of CMP which needs to be recycled. See lipid metabolism for justification. \textbf{ec-code}
 \item Pyrophosphate (diphosphate) hydrolysis reaction. Many energy requiring reactions produce diphosphate that is then rapidly hydrolysed to phosphate providing additional driving force for these reactions. \textbf{Justification?} \textbf{ec-code}
 \item Hydrogenase reaction to reduce NAD$^+$ to NADH and provide reducing equivalents.\textbf{Justification?} \textbf{ec-code}
 \item Acetyl-CoA synthase, several reactions in the pathway produce acetate \textbf{cite them} so this metabolite needs to either be reintegrated into the metabolism or excreted.The most direct and common way to use acetate, is to invest energy and convert it to acetyl coenzyme a so this is the route we propose by introducing \textbf{ec-code}.
 \item ApoACP synthase. Fatty acid synthesis uses ACP as the carrier for chain elongation. Where else does ACP feature? To introduce this into the model. \textbf{ec-code}
\end{itemize}

\subsection*{Cofactor requirements}
\subsection*{Catabolic choices}
\section*{Discussion}
\section*{Methods}
\end{document}
